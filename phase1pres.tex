\documentclass{beamer}
\usetheme{Boadilla}
\usepackage{amsmath, amsfonts, amssymb, mathtools, enumerate, color}
\usepackage{graphicx}
\title[Consensus Protocols in Blockchain]{Consensus Protocols in Blockchain } % The short title appears at the bottom of every slide, the full title is only on the title page
\author[Abhishek Kumar Sinha (173190019)]{\footnotesize IE 797: Phase 1 Seminar\\
by\\
\bf Abhishek Kumar Sinha (173190019)} % Your name

\institute[] % Your institution as it will appear on the bottom of every slide, may be shorthand to save space
{under the guidance of \\
Prof. K.S.Mallikarjuna Rao\\
\medskip
\begin{center}
\includegraphics[scale=0.15]{iit_logo}
\end{center}
Industrial Engineering \& Operations Research \\ % Your institution for the title page
IIT Bombay\\
\medskip
% \textit{sufiyan@iitb.ac.in} % Your email address
}
\date{Oct 15 2018}

\begin{document}

\begin{frame}
\titlepage
\end{frame}


\begin{frame}
\frametitle{Outline}
\tableofcontents
\end{frame}
%%%%%%%%%%%%Section -1 Blockchain Primer%%%%%%%%%%%%%%%%%%%%%%%%%
\section{Introduction to Blockchain}
\begin{frame}{What is a Blockchain?}
It is the following:
\begin{itemize}
    \item \textbf{It is a chain of blocks linked together}
\end{itemize}
\begin{figure}[H]\centering 
\includegraphics[scale = 0.35]{blockchain1.jpg}
\caption{Block-chain}
\end{figure}

\end{frame}
%%%%%%%%%%%%%%%%%%%%%%%%%%%%%%%%%%%%%
\begin{frame}{What is a Blockchain?}
It is the following:
\begin{itemize}
    \item It is a chain of blocks linked together 
    \item \textbf{It is a linear data-structure(each block stores data)}
\end{itemize}
\begin{figure}[H]\centering 
\includegraphics[scale = 0.5]{bc2}
\caption{Block-chain}
\end{figure}

\end{frame}
%%%%%%%%%%%%%%%%%%%%%%%%%%%%%%%%%%%%%
\begin{frame}{What is a Blockchain?}
It is the following:
\begin{itemize}
    \item It is a chain of blocks linked together (apparent)
    \item It is a linear data-structure(each block stores data)
    \item \textbf{It is a distributed ledger(data is in the form of transactions which are stored by everyone on the network)}
\end{itemize}
\begin{figure}[H]\centering 
\includegraphics[scale = 0.25]{bc3}
\caption{Distributed Ledger}
\end{figure}

\end{frame}
%%%%%%%%%%%%%%%%%%%%%%%%%%%%%%%%%%%%%


\begin{frame}{Components of a Block}
n represents integer type \\
hash represents double hash output of SHA-256 hash function
\begin{figure}[H]\centering 
\includegraphics[scale = 0.3]{block}
\caption{Components of a block}
\end{figure}
    
\end{frame}
%%%%%%%%%%%%%%%%%%%%%%%%%%%%%%%%%%%%%
\begin{frame}{SHA 256}
 \begin{itemize}
     \item SHA stands for Secure Hash Algorithm, and 256 signifies the length of output string
     \item Any cryptographic hash function like SHA-256 is supposed to be pre-image resistant, second pre-image resistant and collision resistant
     \item Although a hash function is defined to have an input of arbitrary length, SHA-256 allows $2^{64} - 1$ bits long input
     \item SHA-256 operation is divided into pre-processing and hash computation
 \end{itemize}   
\end{frame}
%%%%%%%%%%%%%%%%%%%%%%%%%%%%%%%%%%%%%
\begin{frame}{SHA 256(Pre-processing)}
    
\end{frame}
%%%%%%%%%%%%%%%%%%%%%%%%%%%%%%%%%%%%%
\begin{frame}{SHA 256(Hash Computation)}
    
\end{frame}
%%%%%%%%%%%%%%%%%%%%%%%%%%%%%%%%%%%%%
\begin{frame}{Merkle Root}
    
\end{frame}
%%%%%%%%%%%%%%%%%%%%%%%%%%%%%%%%%%%%%
\begin{frame}{Nonce}
    
\end{frame}

%%%%%%%%%%%%%%%%%%%%%%bibliography%%%%%%%%%%%%%%%%%%%%%%%%%%%%%%%%%%
\bibliographystyle{plain}
\bibliography{biblist}
\end{document}
