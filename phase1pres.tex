\documentclass{beamer}
\usetheme{Boadilla}
\usepackage{amsmath, amsfonts, amssymb, mathtools, enumerate, color}
\usepackage{graphicx}
\title[Consensus Protocols in Blockchain]{Consensus Protocols in Blockchain } % The short title appears at the bottom of every slide, the full title is only on the title page
\author[Abhishek Kumar Sinha (173190019)]{\footnotesize IE 797: Phase 1 Seminar\\
by\\
\bf Abhishek Kumar Sinha (173190019)} % Your name

\institute[] % Your institution as it will appear on the bottom of every slide, may be shorthand to save space
{under the guidance of \\
Prof. K.S.Mallikarjuna Rao\\
\medskip
\begin{center}
\includegraphics[scale=0.15]{iit_logo}
\end{center}
Industrial Engineering \& Operations Research \\ % Your institution for the title page
IIT Bombay\\
\medskip
% \textit{sufiyan@iitb.ac.in} % Your email address
}
\date{Oct 15 2018}

\begin{document}

\begin{frame}
\titlepage
\end{frame}


\begin{frame}
\frametitle{Outline}
\tableofcontents
\end{frame}
%%%%%%%%%%%%Section -1 Blockchain Primer%%%%%%%%%%%%%%%%%%%%%%%%%
\section{Introduction to Blockchain}
\begin{frame}{What is a Blockchain?}
It is the following:
\begin{itemize}
    \item \textbf{It is a chain of blocks linked together}
\end{itemize}
\begin{figure}[H]\centering 
\includegraphics[scale = 0.35]{blockchain1.jpg}
\caption{Block-chain}
\end{figure}

\end{frame}
%%%%%%%%%%%%%%%%%%%%%%%%%%%%%%%%%%%%%
\begin{frame}{What is a Blockchain?}
It is the following:
\begin{itemize}
    \item It is a chain of blocks linked together 
    \item \textbf{It is a linear data-structure(each block stores data)}
\end{itemize}
\begin{figure}[H]\centering 
\includegraphics[scale = 0.5]{bc2}
\caption{Block-chain}
\end{figure}

\end{frame}
%%%%%%%%%%%%%%%%%%%%%%%%%%%%%%%%%%%%%
\begin{frame}{What is a Blockchain?}
It is the following:
\begin{itemize}
    \item It is a chain of blocks linked together (apparent)
    \item It is a linear data-structure(each block stores data)
    \item \textbf{It is a distributed ledger(data is in the form of transactions which are stored by everyone on the network)}
\end{itemize}
\begin{figure}[H]\centering 
\includegraphics[scale = 0.25]{bc3}
\caption{Distributed Ledger}
\end{figure}

\end{frame}
%%%%%%%%%%%%%%%%%%%%%%%%%%%%%%%%%%%%%%%%%%%%%%%%%%%%%%%%
\begin{frame}{BITCOIN}
\begin{itemize}
    \item It was the pioneer in using block-chain 
    \item User(s) named Satoshi Nagamoto proposed in \cite{nakamoto2008bitcoin} the use of block-chain along with the principles of economics and cryptography for a digital currency named Bitcoin 
    \item Bitcoin is completely digitised and de-centralised crypto-currency (by the users for the users)
\end{itemize}
    
\end{frame}
%%%%%%%%%%%%%%%%%%%%%%%%%%%%%%%%%%%%%


\begin{frame}{Components of a Block}
n represents integer type \\
hash represents double hash output of SHA-256 hash function
\begin{figure}[H]\centering 
\includegraphics[scale = 0.45]{compblock}
\caption{Components of a block}
\end{figure}
    
\end{frame}
%%%%%%%%%%%%%%%%%%%%%%%%%%%%%%%%%%%%%
\begin{frame}{Block Header}
\begin{itemize}
    \item Version - Blockchain version gets updated when more features are added
    \item hashPreviousBlock - Contains double hash output of previous block header
    \item hashMerkle Root - Contains hash of the Merkle Root of the transactions
    \item Nonce - Answer to the cryptographic puzzle
\end{itemize}

 
\end{frame}
%%%%%%%%%%%%%%%%%%%%%%%%%%%%%%%%%%%%%%%%%%%%%%%%%%%%%%%%
\begin{frame}{Merkle Tree}
The hashMerkleRoot field stores the root hash of the Merkle Tree formed using the transactions in the block.
\begin{figure}[H]\centering 
\includegraphics[scale = 0.43]{merkletree.jpg}
\caption{A 3-transaction Merkle Tree}
\end{figure}
\end{frame}
%%%%%%%%%%%%%%%%%%%%%%%%%%%%%%%%%%%%%%%%%%%%

\begin{frame}{List of Transactions}
\begin{itemize}
    \item Number of Transactions include regular and a coinbase transaction
    \item New blocks are made by a process called mining and the miner adds the mining reward by making a coinbase transaction
\end{itemize}


\end{frame}
%%%%%%%%%%%%%%%%%%%%%%%%%%%%%%%%%%%%%%%%%%%%%%%%%%%%%%%%
\section{Bitcoin vs Banks }
\begin{frame}{Bitcoin v Banks}
Bitcoin is a digital currency, so we must compare it with the present system of banking to analyse whether it can form a good alternative. Comparison parameters are as follows:
\begin{itemize}
    \item Identity Management
    \item Fund Transfer
    \item Record Management
    \item Trust
\end{itemize}
 
\end{frame}
%%%%%%%%%%%%%%%%%%%%%%%%%%%%%%%%%%%%%%%%%%%%%%%%%%%%%%%%
\begin{frame}{Identity Management}
\begin{figure}[H]\centering 
\includegraphics[scale = 0.6]{identity.jpg}
\caption{Comparison of identity management between banks and Bitcoin}
\end{figure}
\end{frame}
%%%%%%%%%%%%%%%%%%%%%%%%%%%%%%%%%%%%%%%%%%%%%%%%%%%%%%%%
\begin{frame}{Fund Transfer}
    \begin{itemize}
        \item Banks check for the validation of a transaction initiated and then it proceeds with the bank as the central entity tracking everything for the user and itself
        \item Bitcoin is a de-centralised entity and such a model won't work for it, instead it uses the Unspent Transaction Output(UTXO) model
    \end{itemize}
\end{frame}
%%%%%%%%%%%%%%%%%%%%%%%%%%%%%%%%%%%%%%%%%%%%%%%%%%%%%%%%
\begin{frame}{UTXO Model}
 \begin{figure}[H]\centering 
\includegraphics[scale = 0.5]{utxo1.jpg}
\caption{A has the following UTXOs}
\end{figure}   
\end{frame}
%%%%%%%%%%%%%%%%%%%%%%%%%%%%%%%%%%%%%%%%%%%%%%%%%%%%%%%%
%%%%%%%%%%%%%%%%%%%%%%%%%%%%%%%%%%%%%%%%%%%%%%%%%%%%%%%%
\begin{frame}{UTXO Model}
 \begin{figure}[H]\centering 
\includegraphics[scale = 0.5]{utxo2.jpg}
\caption{B has requested 300 BTC from A}
\end{figure}   
\end{frame}
%%%%%%%%%%%%%%%%%%%%%%%%%%%%%%%%%%%%%%%%%%%%%%%%%%
\begin{frame}{UTXO Model}
\begin{itemize}
    \item This model has been adopted since the transfer of funds is peer to peer and there is no central
authority to verify whether the funds are available or not
\item The complexity of checking transaction validity goes down as the whole history of transaction need not
be verified
\item The complexity of tracking one’s own funds goes up
\end{itemize}
    
\end{frame}
%%%%%%%%%%%%%%%%%%%%%%%%%%%%%%%%%%%%%%%%%%%%%%%%%%
\begin{frame}{Record Management}
 \begin{itemize}
     \item Banks itself keeps updating every account details after every transaction
     \item In Bitcoin every user on the network has a copy of the block-chain(ledger)
     \item Once a transaction is done between users it is stored in a pool called 'MemPool' from where 'Miners' pick up transactions to add to a candidate block 
 \end{itemize} 
 \end{frame}
 
 %%%%%%%%%%%%%%%%%%%%%%%%%%%%%%%%%%%%%%%%%%%%%%%%%%
 
\begin{frame}{Mining}
\begin{itemize}
    \item The process of preparing a new block to append to the block chain is called \textbf{Mining}.Any one on the network can mine new blocks
    \item One has to put in transactions from the mempool in the block along with its coin base transaction and thereafter find the nonce to solve the puzzle
    \item The puzzle is to find the nonce such that when the block header is hashed , its output is below the target hash value set by the network
    \item The hash target is set such that a block is mined every 10 minutes approximately
    \item The nonce is found by brute force
\end{itemize}
 
\end{frame}
%%%%%%%%%%%%%%%%%%%%%%%%%%%%%%%%%%%%%%%%%%%%%%%%%%
\begin{frame}{Mining}
 \begin{figure}[H]\centering 
\includegraphics[scale = 0.4]{mining}
\caption{Mining Procedure}
\end{figure}   
\end{frame}
%%%%%%%%%%%%%%%%%%%%%%%%%%%%%%%%%%%%%%%%%%%%%%%%%%
\begin{frame}{Trust}
\begin{itemize}
    \item Banks are regulated by the government thus we users end up trusting them
    \item Block chain systems are supposed to be trust-less system, where the inherent design is robust against malicious actors
    \item Proof of Work (PoW) consensus ensures that it is practically impossible to game the system
    \item Once a candidate block is mined and is broadcasted to the network, at least 51\% of the users have to agree with the validity of the block for it to be appended, it isn't computationally difficult to check for the validity
\end{itemize}

    
\end{frame}
\section{Consensus Protocols}
%%%%%%%%%%%%%%%%%%%%%%%%%%%%%%%%%%%%%%%%%%%%%%%%%%
\begin{frame}{Consensus Protocols}
    \begin{itemize}
        \item PoW is one of the consensus protocols, different block chains might use other ways to build consensus
        \item Few other protocols include Proof of Stake (PoS), Casper Protocol
        \item They are said to be successful if they solve the  Byzantine Generals Problem as stated in \cite{lamport1982byzantine}
    \end{itemize}
\end{frame}
%%%%%%%%%%%%%%%%%%%%%%%%%%%%%%%%%%%%%%%%%%%%%%%%%%
\begin{frame}{Proof of Stake}
\begin{itemize}
    \item The Proof of stake protocol tries to remove the inefficiency of PoW of power consumption
    \item Instead of miners there are validators who put their money on stake to validate a candidate block who get proportionate reward on successful addition of the block
    \item Users with more money are more likely to become validators with their incentive being not to destabilise the network which could result in de-valuation of their assets
    \item Validators could put stake on multiple chains as there is not sufficient disincentive 
\end{itemize}

    
\end{frame}
%%%%%%%%%%%%%%%%%%%%%%%%%%%%%%%%%%%%%%%%%%%%%%%%%%
\begin{frame}{Casper Protocol}
\begin{itemize}
    \item PoS doesn't punish the validators for putting at stake the money on multiple blocks which could give rise to multiple branches of the chain (Nothing at Stake problem)
    \item This was improved upon by the Casper protocol, by penalising wrong validation and thus losing out your tokens thereafter
\end{itemize}
    
\end{frame}
%%%%%%%%%%%%%%%%%%%%%%%%%%%%%%%%%%%%%%%%%%%%%%%%%%
\section{Smart COntracts}
\begin{frame}{Application to contracts}
    
\end{frame}
%%%%%%%%%%%%%%%%%%%%%%%%%%%%%%%%%%%%%%%%%%%%%%%%%%
\begin{frame}{Frame Title}
    
\end{frame}
%%%%%%%%%%%%%%%%%%%%%%bibliography%%%%%%%%%%%%%%%%%%%%%%%%%%%%%%%%%%
\begin{frame}{References}
 \bibliographystyle{plain}
\bibliography{biblist}   
\end{frame}

\end{document}
